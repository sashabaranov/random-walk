\documentclass{article}
\usepackage{ucs}
\usepackage[utf8x]{inputenc} % Включаем поддержку UTF8
\usepackage[russian]{babel}  % Включаем пакет для поддержки русского языка}
\usepackage{url}
\usepackage{enumerate}

\title{Задача о пьянице}
\author{Баранов Александр}
\date{} 



\begin{document}
 
    \pagestyle{empty} % Use for title page, or use
    \maketitle
    \pagebreak
    
    
        
    \section{Введение}
        Задача о пьянице является частным случаем случайного блуждания \cite{wiki}, в котором условия завершения ведут к определённому конечному состоянию. (Drunkard's walk, a type of a random walk in which termination conditions lead to a biased ending state)

        Постановка задачи: пьяница случайным образом «блуждает» по городу, представленному множеством перпендикулярно пересекающихся улиц. На некоторых перекрёстках города стоят «полицейские». Когда пьяница попадает на перекрёсток с «полицейскими», его блуждание прекращается. Нужно определить с какой вероятностью пьяница дойдёт при заданных параметрах города.

        Данная задача (несмотря на свою классическую постановку) относится к задачам на решение уравнения Шредингера со случайным потенциалом. Такая задача может описывать:
        \begin{enumerate}[a)]
            \item Распространение тока в проводнике или в кристалле с дефектами
            \item Измерение положения частицы и проблему редукции волновой функции
        \end{enumerate}

        Обобщение данной задачи на пространства с произвольной размерностью дает некоторые относительно простые параллели с результатами в теории струн. Кроме того, подобную задачу можно в упрощенной постановке решить аналитически.

    \pagebreak


    \section{Реализация}



    Реализация данной задачи представляет собой программу, на вход которой подаётся(с помощью аргументов командной строки) ряд параметров, приведённых ниже:
    \begin{enumerate}[a)]
        \item Размерность города (флаг -d)
        \item Размер города по каждому измерению (флаг -N)
        \item Количество «полицейских» (флаг -c)
        \item Количество регенераций «полицейских» в городе (флаг -K)
        \item Количество пьяниц на одну регенерацию (флаг -M)
    \end{enumerate}

    Подробнее о использовании этих параметров можно прочитать при вызове программы с флагом -h.

    На выходе выдаются данные в формате: \textit{C P SP T M K}, где:
    \begin{enumerate}[]
        \item C — количество полицейских
        \item P — измеренная вероятность
        \item SP — среднеквадратичное отклонение измеренной вероятности
        \item T — затраченное на выполнение время
        \item M — количество пьяниц на одну регенерацию
        \item K — количество регенераций за всю симуляцию.

    \end{enumerate}

    \pagebreak

    \section{Результаты}
    

    \pagebreak

    \begin{thebibliography}{9}

    \bibitem{wiki}
      Wikipedia — Random-Walk,
      \emph{\url{http://en.wikipedia.org/wiki/Random_walk}}.
      
    \end{thebibliography}

\end{document}